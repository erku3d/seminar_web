\documentclass[12pt]{beamer}

\usepackage{color}
\definecolor{TUCgreen}{HTML}{779a2b}

\usetheme{Madrid}
\usecolortheme[named=TUCgreen]{structure}
%\usecolortheme{lily}	% Farbtheme

\setbeamertemplate{itemize items}[square]
\setbeamertemplate{enumerate items}[square]
\setbeamertemplate{section in toc}[square]
\setbeamerfont{section number projected}{size=\normalsize}


%\setbeamercolor{frametitle}{bg=TUCgreen}
%\setbeamercolor{headline}{bg=red}

\usepackage[utf8]{inputenc}
\usepackage[german]{babel}
\usepackage{amsmath}
\usepackage{amsfonts}
\usepackage{amssymb}
\usepackage{graphicx}
\author[Eric Kunze]{Eric Kunze}
%\title{\includegraphics[height=2cm]{Rust_Logo.eps}\\Rust in the Web}
\title[]{Rust in the Web}
%\setbeamercovered{transparent} 
\setbeamertemplate{navigation symbols}{} 
%\logo{\includegraphics[height=1cm]{Rust_Logo.eps}} 
%\institute{TU Chemnitz} 
\date{04.11.2015} 
\date{} 
%\subject{Zwischenvortrag} 

\usepackage{textpos}

\usepackage{listings}

\usepackage{listings}
\usepackage{color}

\definecolor{mygreen}{rgb}{0,0.6,0}
\definecolor{mygray}{rgb}{0.5,0.5,0.5}
\definecolor{mymauve}{rgb}{0.58,0,0.82}

\lstset{ %
  backgroundcolor=\color{lightgray!50},   % choose the background color; you must add \usepackage{color} or \usepackage{xcolor}
  basicstyle=\scriptsize,        % the size of the fonts that are used for the code
  breakatwhitespace=false,         % sets if automatic breaks should only happen at whitespace
  breaklines=true,                 % sets automatic line breaking
  captionpos=b,                    % sets the caption-position to bottom
  commentstyle=\color{orange},    % comment style
  deletekeywords={...},            % if you want to delete keywords from the given language
  escapeinside={\%*}{*)},          % if you want to add LaTeX within your code
  extendedchars=true,              % lets you use non-ASCII characters; for 8-bits encodings only, does not work with UTF-8
  %frame=single,	                   % adds a frame around the code
  keepspaces=false,                 % keeps spaces in text, useful for keeping indentation of code (possibly needs columns=flexible)
  keywordstyle=\color{mygreen},       % keyword style
  language=c,                 % the language of the code
  otherkeywords={fn,mut,Vec,vec!,let,i32,in,println!},           % if you want to add more keywords to the set
  numbers=left,                    % where to put the line-numbers; possible values are (none, left, right)
  numbersep=3pt,                   % how far the line-numbers are from the code
  numberstyle=\tiny\color{mygray}, % the style that is used for the line-numbers
  rulecolor=\color{black},         % if not set, the frame-color may be changed on line-breaks within not-black text (e.g. comments (green here))
  showspaces=false,                % show spaces everywhere adding particular underscores; it overrides 'showstringspaces'
  showstringspaces=false,          % underline spaces within strings only
  showtabs=false,                  % show tabs within strings adding particular underscores
  stepnumber=1,                    % the step between two line-numbers. If it's 1, each line will be numbered
  stringstyle=\color{mymauve},     % string literal style
  tabsize=2,	                   % sets default tabsize to 2 spaces
  title=\lstname                   % show the filename of files included with \lstinputlisting; also try caption instead of title
}


\usepackage{multirow}


\begin{document}

\begin{frame}{\centerline{Seminar Web Engineering}}
	\begin{center}
		\includegraphics[height=3cm]{Rust_Logo.eps}\\~\\
		\Large \textbf{Rust in the Web}\\~\\
		\normalsize Eric Kunze
	\end{center}
 \end{frame}
 

\addtobeamertemplate{frametitle}{}{%
\begin{textblock*}{10cm}(.74\textwidth,-.95cm)
	\begin{minipage}[c]{0.24\textwidth}
		\footnotesize Rust in the Web 
	\end{minipage}
	\begin{minipage}[c]{0.2\textwidth}
		\includegraphics[height=.8cm]{Rust_Logo.eps}
	\end{minipage}
\end{textblock*}}

\begin{frame}{Inhalt}
\tableofcontents
\end{frame}

%\begin{frame}{Title}
%	Inhalt
%	\begin{itemize}
%	  \item Test
%	  \item Test
%	\end{itemize}
%\begin{block}{Blocktitel}		% normaler Block
%	 Blockinhalt
%\end{block}
% 
%\begin{exampleblock}{Blocktitel}	% Beispielblock
%	 Blockinhalt
%\end{exampleblock}
% 
%\begin{alertblock}{Blocktitel}		% Warnblock
%	 Blockinhalt
%\end{alertblock}
%\end{frame}

\section{Entwicklungsgeschichte}

\begin{frame}{Entwicklungsgeschichte}
	\begin{itemize}
	  \item 2006 persönliches Projekt von Graydon Hoare
	  \item ab 2009 Projekt bei Mozilla
	  \item 15. Mai 2015 Veröffentlichung der Version 1.0
	  \item[]
	  \item Entwicklung einer neuen Browserenging $\rightarrow$ Servo
	 \end{itemize}	 
\end{frame}

\section{Warum Rust?}

\begin{frame}{Warum Rust?}
	\begin{itemize}
	  \item Warum nicht C\texttt{++} oder Java?
	  \item[]
	  \item Rust
	  \begin{itemize}
	      \item schnell
	      \item Kontrolle über das System
	      \begin{itemize}
	          \item minmale Runtime
	          \item keine Garbage Collection
	      \end{itemize}
	      \item sicher	      
	      \item Features von höheren- und funktionalen Programmiersprachen
	  \end{itemize}
	  \end{itemize}
\end{frame}

\section{Was macht Rust sicher?}

\begin{frame}[fragile=singleslide]{Was macht Rust sicher?}
	\begin{itemize}
	    \item Ownership and Borrowing
	\end{itemize}
	\begin{itemize}
	    \item Ownership
	    \begin{itemize}
	        \item jede Ressource hat genau einen Besitzer
	    \end{itemize}    
	\end{itemize}
\begin{center}
\hspace{3pt}
\begin{minipage}[t]{.47\textwidth}
\begin{lstlisting}
fn foo() {	
	let mut y: Vec<i32> 
		= Vec::new();
		
	y.push(4);	
	bar(y);	
	y.push(5); // Compiler Error	
}
\end{lstlisting}				
\end{minipage}
%\hfill
\hspace{3pt}
\begin{minipage}[t]{.47\textwidth}
\begin{lstlisting}
fn bar(x: Vec<i32>) {	
	...
}
\end{lstlisting}				
\end{minipage}
\end{center}
\end{frame}

\begin{frame}[fragile=singleslide]{Was macht Rust sicher?}
	\begin{itemize}
	    \item Borrowing
	    \begin{itemize}
	        \item jede Ressource kann verliehen werden
	    \end{itemize}
	    \item[]
	    \item shared borrow  
	\end{itemize}
\begin{center}
\hspace{3pt}
\begin{minipage}[t]{.47\textwidth}
\begin{lstlisting}
fn foo() {	
	let mut y: Vec<i32> 
		= Vec::new();
		
	y.push(4);	
	bar(&y);	
	y.push(5);	
}
\end{lstlisting}				
\end{minipage}
\hspace{3pt}
\begin{minipage}[t]{.47\textwidth}
\begin{lstlisting}
fn bar(x: &Vec<i32>){
	println!("{}",x[0]);  // Ok	
	x.push(1); // Compiler Error
}
\end{lstlisting}				
\end{minipage}
\end{center}
\end{frame}


\begin{frame}[fragile=singleslide]{Was macht Rust sicher?}
	\begin{itemize}
	    \item mutable borrow  
	    \begin{itemize}
	        \item nur eine Referenz auf eine Ressource und diese hat nur einen Besitzer
	    \end{itemize}
	\end{itemize}
\begin{center}
\hspace{3pt}
\begin{minipage}[t]{.48\textwidth}
\begin{lstlisting}
fn foo() {	
	let mut y 
		= vec![1, 2, 3, 4, 5];
	let mut x: Vec<i32> 
		= Vec::new();
		
	bar(&y, &mut x); // Ok
	bar(&y, &mut y); 
		// Compiler Error
}
\end{lstlisting}				
\end{minipage}
\hspace{3pt}
\begin{minipage}[t]{.47\textwidth}
\begin{lstlisting}
fn bar(y: &Vec<i32>, 
		   x: &mut Vec<i32>){
	
	for v in y{
		x.push(*v);
	}
}
\end{lstlisting}				
\end{minipage}
\end{center}
\end{frame}

\begin{frame}[fragile=singleslide]{Was macht Rust sicher?}
\begin{itemize}
    \item Aliasing und Mutation zur gleichen Zeit wird verhindert
   	\item Bsp. C\texttt{++}   	
\end{itemize}
\begin{center}
\hspace{3pt}
\begin{minipage}[t]{.47\textwidth}
\begin{lstlisting}[language=C++,otherkeywords={}]
void foo(){ 
	int *y = new int[10];	
	for(int i=0;i<10;i++)
		y[i] = i;

	int *x = &y[9];
	y=bar(y);	
	delete[] y;		
	cout<<*x<<endl;
}
\end{lstlisting}				
\end{minipage}
\hspace{3pt}
\begin{minipage}[t]{.47\textwidth}
\begin{lstlisting}[language=C++,otherkeywords={}]
int* bar(int *v){
	delete[] v;	
	v = new int[5]	
	for(int i=0;i<10;i++){
		v[i] = i*2;
	}	
	return v;
}
\end{lstlisting}				
\end{minipage}
\end{center}
\end{frame}

\section{Rust in the Web}

\begin{frame}{Rust in the Web}

\begin{minipage}[t]{.45\textwidth}
\begin{itemize}
  \item HTTP-Server
  \begin{itemize}
    \item Hyper
    \item tiny-http
  \end{itemize}
  \item []
  \item HTTP client
  \begin{itemize}
    \item Hyper
    \item curl-rust
  \end{itemize} 
  \item []
  \item Database drivers
  \begin{itemize}
    \item rust-postgres
    \item redis-rs
  \end{itemize}
\end{itemize}
\end{minipage}
\hfill
\begin{minipage}[t]{.45\textwidth}
\begin{itemize}
\item Frameworks
  \begin{itemize}
    \item Iron
    \item Conduit
    \item rustful
    \item Nickel
  \end{itemize}  
\end{itemize}
\end{minipage}

\end{frame}

\section{Iron}
\begin{frame}{Iron}
\begin{itemize}
    \item Webframework
    \item basiert auf Hyper
    \item hochgradig Nebenläufig
    \item keine unnötigen Features im Basisframework, aber leicht erweiterbar
    \item bietet Infrastruktur, um das Framework an individuelle Bedürfnisse anzupassen
\end{itemize}

\end{frame}


\section{Demo}
\begin{frame}{Demo}

\end{frame}

\end{document}